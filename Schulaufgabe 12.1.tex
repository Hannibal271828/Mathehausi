%Dokumentklasse
\documentclass[a4paper,12pt,pointlessnumbers]{scrreprt}
\usepackage[left=2.5cm,right =25mm, bottom =2.5cm, top=25mm, bindingoffset=0mm]{geometry}
%\usepackage[onehalfspacing]{setspace}
% ============= Packages =============

% Dokumentinformationen
\usepackage[
	pdftitle={Einblick in höhere Dimensionen},
	pdfsubject={},
	pdfauthor={Hanifah Muhammad},
	pdfkeywords={},	
	%Links nicht einrahmen
	hidelinks
]{hyperref}



% Standard Packages
\usepackage[utf8]{inputenc}
\usepackage[ngerman]{babel}
\usepackage[T1]{fontenc}
\usepackage{graphicx}
\graphicspath{{img/}}
\usepackage{fancyhdr}
\usepackage{lmodern}
\usepackage{color}
\usepackage{tkz-euclide}
\usepackage[nolist]{acronym} %Abkürzungen
\usepackage{sidecap}
\usepackage{subcaption}

% zusätzliche Schriftzeichen der American Mathematical Society
\usepackage{amsfonts}
\usepackage{amsmath}
	\DeclareMathOperator{\Abb}{Abb}
\usepackage{amsthm}
\usepackage{stmaryrd}
\usepackage{mathtools}

% Gänsefüßchen
\usepackage{csquotes}
\MakeOuterQuote{"}

%Für Vektorpraphiken
\usepackage{transparent}
\graphicspath{{Bilder/}}



%================== amsthm ====================
\theoremstyle{definition}
\newtheorem{definition}{Definition}[section]
\newtheorem{preremark}{Definition}
\newenvironment{Definition}{\begin{preremark}\upshape}{\end{preremark}}
\newtheorem{example}[definition]{Beispiel}
\newtheorem{bem}[definition]{Bemerkung}
\newtheorem{theorem}[definition]{Theorem}
\newtheorem{Satz}[definition]{Satz}
\newtheorem{Corollar}[definition]{Corollar}
\newtheorem{prop}[definition]{Proposition}
\newtheorem{Lemma}[definition]{Lemma}



%nicht einrücken nach Absatz
%\setlength{\parindent}{0pt}


% ============= Kopf- und Fußzeile =============
%\pagestyle{fancy}
%
\lhead{}
\chead{}
\rhead{\slshape \leftmark}
%%
\lfoot{}
\cfoot{\thepage}
\rfoot{}
%%
\renewcommand{\headrulewidth}{0.4pt}
\renewcommand{\footrulewidth}{0pt}
\renewcommand{\figurename}{Abb.}
% ============= Package Einstellungen & Sonstiges ============= 
%Besondere Trennungen
\hyphenation{De-zi-mal-tren-nung}

% ============= Modifikationen von Aufzählungen ===============
\renewcommand\theenumi{\roman{enumi}}
\renewcommand\labelenumi{(\theenumi)}

\usepackage[nolist]{acronym}
\begin{acronym}
 \acro{VR}{Vektorraum}
 \acrodefplural{VR}[VRs]{Vektorräume}
\end{acronym}

\begin{document}
\title{Zusammenfassung für die Schulaufgabe}
\author{Hanifah Muhammad}

\tableofcontents
\chapter{Die Integralrechnung}
\section{Das Integral}
Berechnung der Untersumme:
\begin{align*}
U_n=A_1+...+A_n <\; \text{tatsächliche Fläche}
\end{align*}
Berechnung der Obersumme:
\begin{align*}
O_n=A_1+...+A_n >\; \text{tatsächliche Fläche}
\end{align*}
\begin{definition}
Sei $f(x)$ eine Funktion mit $f(x)\geq 0$ im Intervall $[a;b]$. Dann nennt man den gemeinsamen Grenzwert von Unter- und Obersumme 
\begin{align*}
\lim\limits_{n \to \infty} U_n = \lim\limits_{n \to \infty} O_n 
\end{align*}
das Integral der Funktion $f$ zwischen den Grenzen $a$ und $b$.
Man schreibt hierfür \[\int_a^bf(x)dx\text{.}\]
(Lies: "Das Integral von $f(x) dx$ von $a$ bis $b$.")
Weitere Begriffe
\begin{itemize}
\item $b$ obere Grenze
\item $a$ untere Grenze
\item $f(x)$ Integrand
\item $f(x)dx$ Integrationsvariable 
\end{itemize}
Vereinbarung: Sei $a<b$.
\begin{align*}
\int_b^a f(x) dx \vcentcolon= - \int_a^b f(x) dx
\end{align*}
\end{definition}
\newpage
\section{Die Flächenbilanz}
bisher: $f(x) \geq 0 \;\forall x \in [a;b]$
\\ zunächst: $f(x) \leq 0 \;\forall x \in [a;b]$
\\ Ist die Integrandenfunktion im Integrationsintervall negativ, gibt das Integral \underline{nicht} die Fläche an, sondern ihr negatives.
\[\int_a^b f(x) dx < 0 \Rightarrow A = \Bigl|\int_a^b f(x) dx\Bigr| = - \int_a^b f(x) dx 
\]
Allgemein: Sei $f$ eine Funktion, die im Intervall $[a;b]$ definiert ist. Das Integral $\int_a^b f(x) dx$ gibt die \textbf{Flächenbilanz}\footnote{Summe der über der $x$-Achse Flächeninhalte $+$ Summe der unter der $x$-Achse gelegenen Flächeninhalte} an.

\begin{example}
\begin{align}
&\int_{-1}^1 x dx = 0
\\ &\int_0^{10}(-\frac{1}{2}x+2) dx =-5
\end{align}
\end{example}

\section{Die Integralfunktion}
bisher: Integrale mit fester unterer und oberer Grenze
\begin{align*}
\int_b^a f(x) dx\;\text{(\textbf{bestimmtes Integral})}
\end{align*}
jetzt: Untere Grenze $a$ bleibt fest; obere wird zur Variablen $x$. Man erhält eine Funktion:
\begin{align*}
I_a (x)=\int_a^x f(t) dt \;\text{(\textbf{Integralfunktion})}
\end{align*}

\fbox{Jede Integralfunktion hat mindestens eine Nullstelle!}

\newpage
\section{Der Hauptsatz der Differential- und Integralrechnung}
Vermutung: Die Integralfunktion ist eine Stammfunktion der Integrandenfunktion.
\begin{theorem}[\emph{Hauptsatz der Differential- und Integralrechnung}]
Sei $f:t\mapsto f(t)$ im Intervall $[a;b]$ definiert. Dann gilt für die Integralfunktion $I_a (x)=\int_a^x f(t) dt$:
\begin{align*}
 I_a' (x)=f(x) \;\forall x \in [a;b]
\end{align*}
Die Integralfunktion $I_a$ ist eine Stammfunktion der Integrandenfunktion $f$.
\end{theorem}
\begin{proof}
\[A=I_a(x+h)-I_a(x)\]
$m_h$: kleinster Funktionswert in \([x;x+h]\)
\\ $M_h$: größter Funktionswert in \([x;x+h]\)
\\Dann gilt:
\begin{align*}
    h \cdot m_h < I_a(x+h)-I_a(x) < h \cdot M_h
    \\ m_h < \frac{I_a(x+h)-I_a(x)}{h}< M_h
\end{align*}
Für \(h\rightarrow 0 \Rightarrow m_h \rightarrow f(x); M_h \rightarrow f(x)\)
\begin{align*}
    \lim\limits_{h \to 0}   \frac{I_a(x+h)-I_a(x)}{h} = f(x)
    \Rightarrow I_a'(x)=f(x)
\end{align*}
\end{proof}
$\Rightarrow I_a(x)$ ist \textsc{eine} (bestimmte) Stammfunktion von $f(x)$. Sei nun $F(x)$ eine beliebige Stammfunktion von $f(x)$.
\begin{align}
    \Rightarrow I_a(x)= F(x)+c \label{int}
\end{align}
Wegen $I_a(a)=0$ 
In \ref{int} und $x$ für $a$ einsetzen:
\begin{align*}
    I_a(a)&=F(a)+c
    \\ 0&=F(a)+c
    \\ c&=-F(a)
    \Rightarrow I_a(x) = F(x)-F(a)
\end{align*}
D.h., dass die Stammfunktion solange in $y$-Richtung verschoben werden muss, sodass die Nullstelle stimmt.
Insbesondere:
\\ 
\begin{center}
\fbox{$I_a(b)=\int_a^b f(t) dt = F(b)-F(a)$}    
\end{center}
    Neue Schreibweise: $\int_a^b f(t) dt=[F(t)]_a^b$ 
\newpage
\begin{example}
\begin{itemize}
\item \(\int_{-1}^3 (x^2-1) dx = \bigl[\frac{1}{3}x^3-x\bigr]_{-1}^{3}= \bigl(\frac{1}{3}\cdot 3^3-3\bigr)-\bigl(\frac{1}{3}\cdot(-1)^3-(-1)\bigr)=
\frac{16}{3}\)
\item \(\int_{-1}^3 (x^2-1) dx = \bigl[\frac{1}{3}x^3-x+c\bigr]_{-1}^{3}= \bigl(\frac{1}{3}\cdot 3^3-3+c\bigr)-\bigl(\frac{1}{3}\cdot(-1)^3-(-1)+c\bigr)=
\frac{16}{3}\)
\item \(\int_{-1}^x (x^2-1) dx = \bigl[\frac{1}{3}x^3-x\bigr]_{-1}^{x}= \bigl(\frac{1}{3}\cdot x^3-x\bigr)-\bigl(\frac{1}{3}\cdot(-1)^3-(-1)\bigr)=\frac{1}{3}\cdot x^3-x-\frac{2}{3}
\)
\end{itemize}
\end{example}

\begin{Corollar}
\begin{enumerate}
\item $\int_a^a f(t) dt = 0$ 
\item $\int_b^a f(t) dt = - \int_a^b f(t) dt$ 
\item $\int_b^a f(t) dt = \int_a^c f(t) dt + \int_c^b f(t) dt$
\end{enumerate}
\end{Corollar}

\begin{proof}
\begin{enumerate}
\item $F(a)-F(a)=0$
\item $F(a)-F(b)=-(F(b)-F(a))$
\item $F(c)-F(a)+F(b)-F(c)=F(b)-F(a)$
\end{enumerate}
\end{proof}

\section{Die Stammfunktion}
neue Schreibweise:
Das \textsc{unbestimmte Integral} $\int f(x) dx$ wir als Symbol für die Menge aller Stammfunktionen einer Funktion $f(x)$ verwendet.
\begin{example}
\begin{align*}
\int 2x^3 dx = \bigl\{ \frac{1}{2} x^4 +c \mid c \in \mathbb{R}\bigr\}
\end{align*}
\end{example}
Stammfunktionen zu einfachen Funktionen:
\begin{center}
    \begin{tabular}{| l | l | l |}
    \hline
    $f(x)$ & $F(x)$ & Merkhilfe \\ \hline
    $x^r (r\neq -1)$&$\frac{1}{r+1}x^r+1$&$\int x^r dx = \frac{1}{r+1}x^r+1 +c (r\neq -1)$     \\ \hline 
    $\frac{1}{x}$&$\ln|x|$& $\int\frac{1}{x}dx= \ln|x|+c$ \\ \hline
    $\sin x$ &$-\cos x$& $\int \sin x dx= -\cos x + c$ \\ \hline
    $\cos x$ & $\sin x$ & $\int \cos x dx= \sin x +c $ \\ \hline
    $e^x$ & $e^x$ & $\int e^x dx= e^x+c$ \\ \hline
    $\ln x$&$x\ln x -x$&$\int \ln x dx = x - \ln x +c$\\ \hline
    \end{tabular}
\end{center}
\begin{Satz}[\emph{Rechenregeln mit Stammfunktionen und Integralen}]
Seien $G$ und $H$ jeweils Stammfunktionen zu $g$ und $h$, $k\in \mathbb{R}$. Dann gilt:
\begin{enumerate}
\item Sei $f(x)=g(x)+h(x)$, so ist $F(x)=G(x)+H(x)$ Stammfunktion von $f(x)$, denn $F'(x)=G'(x)+H'(x)$ (Summenregel).
\[\Rightarrow \int_a^b (g(x)+h(x)) dx= \int_a^b g(x) dx + \int_a^b h(x) dx\]
\item Sei $f(x)=k \cdot g(x)$, so ist $F(x)=k\cdot G(x)$, denn $F'(x)=k\cdot G'(x)$ (multiplikativer Faktor).
\end{enumerate}
\end{Satz}

\section{Die Flächenberechnung mit dem Integral}
\begin{enumerate}
\item (\emph{Fläche zwischen Graph und $x$-Achse}). Beachte, die Flächen zwischen den ggf. vorhandenen Nullstellen zu berechen, da man sonst die Flächenbilanz errechnet.
\item (\emph{Fläche zwischen zwei Graphen}). Seien alle $s_n$ Schnittstellen von $f(x)$ mit $g(x)$. Wenn bekannt ist, dass $f(x)\geq g(x)$, $f(x)$ als Minuend, analog für $g(x)$. Sicherheitshalber Betragsstriche um alle Summanden setzen.
\[A=\bigl|\int_a^{s_1} (f(x)-g(x))dx \bigr|+ 
\bigl|\int_{s_1}^{s_2} (f(x)-g(x))dx \bigr|+ ... + \bigl|\int_{s_{n-1}}^{s_n} (f(x)-g(x))dx\bigr|\]
\end{enumerate}

\section{Die speziellen Integrale}
Erinnerung: Sind $F(x)$ und $G(x)$ Stammfunktionen von $f(x)$ und $g(x)$, so ist $F(x) \cdot G(x)$ keine Stammfunktion von $f(x) \cdot g(x)$. Aber in einigen Sonderfällen lassen sich dennoch Stammfunktionen angeben. 
\begin{enumerate}
\item (\textit{Die $e$-Funktion als ein Faktor}). \[ \int e^{f(x)}\cdot f'(x) dx= e^{f(x)} + c\]
\begin{example}
\[\int  e^{3x^2}\cdot 6x dx= e^{3x^2} + c\]
\end{example} 
\item (\textit{Im Zähler steht die Ableitung des Nenners}).
\[\int \frac{f'(x)}{f(x)}=ln|f(x)|+c\]
\begin{example}
\[\int \frac{2x}{x^2+3}=ln|x^2+3|+c\]
\end{example}
\item (\textit{Verkettungen}). Eine Stammfunktion von $f(x) = (u \circ v) = (x)=u(v(x))$ lässt genau dann angeben, wenn die innere Funktion $v(x)=ax+b$ linear ist:
\[\int f(ax+b)= \frac{1}{a}\cdot F(ax+b)+c\]
\begin{example}
\[\int \cos (2x+3)= \frac{1}{2} \sin(2x+3)+c\]
\end{example}
\end{enumerate}

\section{Die uneigentlichen Integrale}
Ins Unendliche reichende Flächen \underline{können} einen endlichen Flächeninhalt haben. Man schreibt hierfür:
\begin{definition}[\emph{Uneigentliche Integrale}]
\begin{align*}
\int_a^z f(x) dx \vcentcolon= \lim\limits_{h \to z} \int_a^h f(x) dx = \lim\limits_{h \to z} [F(x)]_a^h =\lim\limits_{h \to z} F(h)-F(a)=F(a)\in \mathbb{R}
\end{align*}
\end{definition}
\chapter{Weitere Eigenschaften von Funktionen und deren Graphen}



\section{Die Wendepunkte und Art der Extrema}
\begin{definition}[\emph{Die Höhere Ableitung}]
Ist die Ableitung einer Funktion $f$ differenzierbar, so erhält man durch Ableiten von $f'$ die zweite Ableitung $f''$. Analog können höhere Ableitungen definiert werden.
\end{definition}
\begin{definition}[\emph{Die Krümmung von Graphen}]
\item $G_f$ ist linksgekrümmt $\Leftrightarrow f''(x)>0$
\item $G_f$ ist rechtsgekrümmt $\Leftrightarrow f''(x)<0$
\end{definition}

\begin{definition}[\emph{Wendestellen/-punkte}]
Eine Stelle $x_0$, an der eine differenzierbare Funktion $f$ ihre Krümmung (von links nach rechts oder umgekehrt) ändert, heißt \textsc{Wendestelle}. Der zugehörige Punkt $(x_0|f(x_0)$ heißt \textsc{Wendepunkt}.
\end{definition}

\begin{definition}[\emph{Terrassenpunkt}]
Ein Wendepunkt mit waagrechter tangente ist ein \textsc{Terrassenpunkt}.
\end{definition}

\begin{Satz}[\emph{Kriterium für Wendestellen}]
\item notwendig: $f''(x)=0$
\item hinreichend: $f''(x)$ hat einen Vorzeichenwechsel (oder $f'''(x)\neq 0$)
\end{Satz}

\begin{Satz}[\emph{Kriterium für Extrema}]
\item notwendig: $f'(x)=0$
\item hinreichend: Vorzeichenwechsel von $f'$ von $-$ nach $+ \Rightarrow$ Minumum; Vorzeichenwechsel von $f'$ von $+$ nach $- \Rightarrow$ Maximum
\\ oder: $f''(x_0)>0 \Rightarrow$ Maxmimum; $f''(x_0)<0 \Rightarrow$ Minumum 

\end{Satz}
\end{document}

